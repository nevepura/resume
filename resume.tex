\documentclass{article}

\usepackage[T1]{fontenc}
\usepackage[utf8]{inputenc}
\usepackage[italian]{babel}

\usepackage{hyperref}
\hypersetup{colorlinks,urlcolor=blue}

\begin{document}
\section{Test}
	\begin{minipage}{0.4 \textwidth}
		contenuto...
	\end{minipage}

\section*{Info e Contatti}
	\begin{tabular}[t]{l  l}
		nato il & 25 agosto 1993\\
		residenza & Arzergrande, PD, Italia \\
		email & \href{mailto:federico.bicciato.93@gmail.com}{federico.bicciato.93@gmail.com}\\
		LinkedIn & \href{https://bit.ly/35jFr8N}{https://bit.ly/35jFr8N}\\
		cellulare  & +39 348 9202662
	\end{tabular}

\section*{Lavoro}
\subsection*{Sviluppatore software, area deep learning}
presso : RiskApp S.r.l\\
{Giugno 2018 - Agosto 2018}
{}\\
Ho sviluppato un algoritmo di apprendimento automatico per l'individuazione di tabelle nei documenti. Ciò ha incluso: ricerca di dataset, parsing di annotazioni con Python, allenamento e tuning di reti neurali con la suite di TensorFlow, Microsoft Azure per lavorare in remoto.\\
Link: \url{https://github.com/nevepura/TableTrainNet}

{Conselve, PD}

\subsection*{Ground handler}
{Italian Ho.St}\\
{Maggio 2018 - Dicembre 2018}\\
{Venezia Terminal Passeggeri}\\
Il lavoro consisteva nella gestione del flusso di passeggeri in transito nelle navi da crociera; mi sono occupato in particolare di accoglienza, fornitura di informazioni su luoghi e mezzi di trasporto, controllo documentale e assistenza ai passeggeri con disabilità.
Il lavoro mi ha permesso di parlare in inglese e spagnolo, di collaborare con le molte persone e compagnie in opera al porto, di apprendere nozioni di logistica.


\section*{Progetti}
\subsection*{Soldino}
Sito web di compravendita di beni su blockchain Ethereum. Frontend in React/Redux, backend in Solidity e Truffle. Ho realizzato script in Python per prelevare le metriche, documentazione in Latex, presentazioni con Google Sheets e Google Slides. Progetto di gruppo.\\
Documenti: \url{https://github.com/8LabSolutions/Soldino_Docs}\\
Codice: \url{https://github.com/8LabSolutions/Soldino}\\
Demo: \url{http://soldino.surge.sh/}

\subsection*{WeBotanic}
Sito web di un orto botanico, scritto in xhtml, css, javascript, php. Ho realizzato la struttura del sito in html e php, e il meccanismo di login in php. Progetto di gruppo.\\
Link: \url{https://github.com/nevepura/WeBotanic}\\
Demo: \url{https://webotanic.000webhostapp.com/index.php}

%\subsection{DroneSim}

\section*{Istruzione}

Laurea triennale in Informatica {Settembre 2014 - Settembre 2019}{Universit\`a degli studi di Padova}{Voto 98/110.}

{Corso di laurea in Ingegneria Edile-Architettura}{Settembre 2012 - Giugno 2014}{Universit\`a degli studi di Padova}{Frequentati i primi due anni, dopo i quali sono passato ad Informatica.}

{Maturità di liceo scientifico, indirizzo PNI}{Settembre 2007 - Luglio 2012}{Istituto Statale di Istruzione Superiore A. Einstein}{Voto 66/100.} 

\section*{Competenze}
OOP: C++, Qt, Java\\
Scripting: Python3\\
Web: Html, Css\\
Lingue: Italiano, Inglese, Spagnolo, Giapponese.\\

\end{document}